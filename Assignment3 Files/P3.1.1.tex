We start by defining two sets : 
\begin{gather*}
    \mathcal{P}(\mathbb{F}_q, n,r) := \{(p(X))^r\in \mathbb{F}_q[X] : \deg{p}=n, \text{$p$ is monic}\} \\
    \mathcal{Q}(\mathbb{F}_q,n) := \{p(X) \in \mathbb{F}_q[X] : \deg{p}=n, \text{$p$ is monic and square free}\}
\end{gather*}

Note that for the given problem we have : 
\[\operatorname{card}(\mathcal{P}(\mathbb{F}_q, n,1))=q^n, \quad \operatorname{card}(\mathcal{Q}(\mathbb{F}_q,n)) = s_n.\]
\subsection{Part (a)}
Let $p(X)\in \mathbb{F}_q[X]$ be a generic monic polynomial of degree $n$. Since $\mathbb{F}_q[X]$ is a UFD, we can uniquely realise the following : 
\[p(X) = \displaystyle\prod_{i=1}^\eta g_{i}(X)^{\nu(i)}\]
where $g_i$'s are irreducible polynomials over $\mathbb{F}_q[X]$.

We can further decompose this as follows : 
\[p(X) = \displaystyle\prod_{i : 2\mid\nu(i)}g_i(X)^{\nu(i)}\displaystyle\prod_{i : 2\nmid\nu(i)}g_i(X)^{\nu(i)}.\]
The second part of the product can be further decomposed as follows : 
\[p(X) = \mathcolor{teal}{\left(\displaystyle\prod_{i : 2\mid\nu(i)}g_i(X)^{\nu(i)/2}\displaystyle\prod_{i : 2\nmid\nu(i)}g_i(X)^{(\nu(i)-1)/2}\right)^2}\mathcolor{orange}{\displaystyle\prod_{i : 2\nmid\nu(i)}g_i(X)}.\]
If the \textcolor{teal}{teal} product is of degree $2k$, then the \textcolor{orange}{orange} product results in a polynomial of degree $n-2k$.

We can define the following map : 
\begin{gather*}\Psi : \mathcal{P}(\mathbb{F}_q, n,1) \to \coprod_{0\leq 2k\leq n} \mathcal{P}(\mathbb{F}_q, k,2)\times \mathcal{Q}(\mathbb{F}_q, n-2k) \\
p(X) \mapsto \left(\left(\displaystyle\prod_{i : 2\mid\nu(i)}g_i(X)^{\nu(i)/2}\displaystyle\prod_{i : 2\nmid\nu(i)}g_i(X)^{(\nu(i)-1)/2}\right)^2,\displaystyle\prod_{i : 2\nmid\nu(i)}g_i(X)\right).
\end{gather*}
This map is clearly well defined. We note that if $\Psi(p(X)) = (p_1(X), p_2(X))$, then $p(X) = p_1(X)p_2(X)$. If $\Psi(p(X)) = \Psi(q(X))$, then $(p_1(X), p_2(X)) = (q_1(X), q_2(X))$ or $p(X) = p_1(X)p_2(X)=q_1(X)q_2(X) = q(X)$. Therefore, $\Psi$ is injective. $\Psi$ is clearly surjective because for every $(p_1(X), p_2(X))$, $\exists\ p(X) = p_1(X)p_2(X)$, $\deg{p}= \deg{p_1}+\deg{p_2} = 2k + n-2k = n$, such that $\Psi(p(X)) = (p_1(X), p_2(X))$. Hence, $\Psi$ is a bijection.

Therefore, the cardinalities of the domain and range of $\Psi$ must be the same.
\begin{align*}
    q^n &=\operatorname{card}(\mathcal{P}(\mathbb{F}_q, n,1)) = \operatorname{card}\left(\coprod_{0\leq 2k\leq n} \mathcal{P}(\mathbb{F}_q, k,2)\times \mathcal{Q}(\mathbb{F}_q, n-2k)\right) \\
    &= \sum_{0\leq 2k\leq n} \operatorname{card}(\mathcal{P}(\mathbb{F}_q, k,2)\times \mathcal{Q}(\mathbb{F}_q, n-2k)) \\
    &= \sum_{0\leq 2k\leq n} \operatorname{card}(\mathcal{P}(\mathbb{F}_q, k,2))\cdot\operatorname{card}(\mathcal{Q}(\mathbb{F}_q, n-2k)) \\
    &= \sum_{0\leq 2k\leq n} \operatorname{card}(\mathcal{P}(\mathbb{F}_q, k,1))\cdot\operatorname{card}(\mathcal{Q}(\mathbb{F}_q, n-2k)) \\
    &= \sum_{0\leq 2k\leq n} q^k \cdot s_{n-2k}\\
\end{align*}
as required.
\subsection{Part (b)}
From Part $(a)$, we have : 
\[q^n = \sum_{0\leq 2k\leq n} q^k \cdot s_{n-2k},\]
Hence, we can write : 
\[\sum_{n=0}^{\infty} q^n t^n = \sum_{n=0}^{\infty} \sum_{0\leq 2k\leq n} q^k s_{n-2k}t^n.\]
We note the pattern in $\sum_{n=0}^{\infty} \sum_{0\leq 2k\leq n} q^k s_{n-2k}t^n = $ 
\begin{align*}
     &= \mathcolor{teal}{q^0 s_0 t^0} \\
    &+ \mathcolor{orange}{q^0 s_1 t^1} \\
    &+ \mathcolor{red}{q^0 s_2 t^2} + \mathcolor{teal}{q^1 s_0 t^2} \\
    &+ \mathcolor{purple}{q^0 s_3 t^3} + \mathcolor{orange}{q^1 s_1 t^3}  \\
    &+ \mathcolor{cyan}{q^0 s_4 t^4} + \mathcolor{red}{q^1 s_2 t^4} + \mathcolor{teal}{q^2 s_0 t^4} \\ 
    &+ q^0 s_5 t^5 + \mathcolor{purple}{q^1 s_3 t^5} + \mathcolor{orange}{q^2 s_1 t^5} \\
    &+ q^0 s_6 t^6 + \mathcolor{cyan}{q^1 s_4 t^6} + \mathcolor{red}{q^2 s_2 t^6} + \mathcolor{teal}{q^3 s_0 t^6} \\
    &\vdots
\end{align*}
Hence, we can rewrite the sum as : 
\begin{align*}
    \sum_{n=0}^{\infty} q^n t^n = \sum_{n=0}^{\infty} \sum_{0\leq 2k\leq n} q^k s_{n-2k}t^n = 
    \mathcolor{teal}{s_0 \sum_{n=0}^{\infty}(qt^2)^n} + \mathcolor{orange}{s_1 t \sum_{n=0}^{\infty}(qt^2)^n} + \ldots = \left(\sum_{n=0}^{\infty}(qt^2)^n\right)\left(\sum_{n=0}^{\infty}t^n s_n\right)
\end{align*}
Therefore, 
\[\sum_{n=0}^{\infty}t^k s_k = (1-qt^2)\sum_{n=0}^{\infty} q^n t^n = 1 + qt + \sum_{n\geq 2} (q^n - q^{n-1})t^n.\]
Comparing coefficients on both sides, we get : 
\[s_n = \begin{cases} q^n & n=0,1 \\ q^n - q^{n-1} & n\geq 2\end{cases}\tag{$n\geq 2$}.\]

The probability that a random polynomial in $\mathbb{F}_q[X]$ of degree $n$ being square free is therefore, 

\[\Pr[p(X)\in \mathcal{Q}(\mathbb{F}_q,n) | p(X) \in \mathcal{P}(\mathbb{F}_q, n,1)] = \dfrac{\operatorname{card}(\mathcal{Q}(\mathbb{F}_q,n))}{\operatorname{card}(\mathcal{P}(\mathbb{F}_q, n,1))} = \dfrac{s_n}{q^n} = \begin{cases}1 & n=0,1 \\ 1-\frac{1}{q} & n\geq 2\end{cases}.\]



