\begin{thisnote}
\begin{center}
\begin{minipage}{.7\linewidth}
\begin{algorithm}[H]
    \SetAlgoLined
    \KwData{\(f,g\in \mathbb{F}[X]\), where $\mathbb{F}$ is a field, $\deg{f}\geq \deg{g}, g\neq 0$.}
    \KwResult{\(l\in \N,r_i\), for $0\leq i\leq l+1$ and $q_i\in\mathbb{F}[X]$ for $1\leq i\leq l$.}

    $r_0\leftarrow f$, $r_1\leftarrow g$\;
    $i\leftarrow 1$\;
    \While{$r_i\neq 0$}{
        $q_i \leftarrow r_{i-1}\ \operatorname{quo}\ r_i$\;
        $r_{i+1} \leftarrow r_{i-1}-q_i r_i$\;
        $i\leftarrow i+1$\;
    }
    $l\leftarrow i-1$\;
    \KwRet{$l,r_i$ for $0\leq i\leq l+1$, and $q_i$ for $1\leq i\leq l$.}
    \caption{\textsc{Traditional Extended Euclidean Algorithm}}
\end{algorithm}
\end{minipage}
\end{center}
\end{thisnote}
\begin{thisnote1}
Let $\mathscr{P}(\mathbb{F}, d)$ be the set $\{p(X) \in \mathbb{F}[X]\setminus\{0\} ,\deg{p}=d\}$.

\textbf{Lemma : } $|\mathscr{P}(\mathbb{F}_q, d)| = (q-1)q^d$.

\textit{Proof. } Consider the following map : 
\begin{gather*}
    \phi : \mathscr{P}(\mathbb{F}_q, d) \to \mathbb{F}_q^* \times \mathbb{F}_q^{d} \\
    p(X) = a_d X^d + a_{d-1}X^{d-1}+\ldots a_0 \mapsto (a_d, a_{d-1},\ldots, a_{0}).
\end{gather*}
This map is well defined because since $p(X)$ is of degree $d$, $a_d\neq 0$. It is clearly injective and surjective. Therefore, 
\[|\mathscr{P}(\mathbb{F}_q, d)| = |\mathbb{F}_q^*\times \mathbb{F}_q^d| = |\mathbb{F}_q^*||\mathbb{F}_q|^d = (q-1)q^d.\]
\hfill $\blacksquare$
\end{thisnote1}
% \begin{center}
% \begin{minipage}{.7\linewidth}
% \begin{algorithm}[H]
%     \SetAlgoLined
%     \KwData{\(f,g\in \mathbb{F}[X]\), where $\mathbb{F}$ is a field, $\deg{f}\geq \deg{g}, g\neq 0$.}
%     \KwResult{\(l\in \N,r_i,s_i,t_i\), for $0\leq i\leq l+1$ and $q_i\in\mathbb{F}[X]$ for $1\leq i\leq l$.}

%     $r_0\leftarrow f,\quad s_0\leftarrow 1,\quad t_0\leftarrow 0$\;
%     $r_1\leftarrow g,\quad s_1\leftarrow 0,\quad t_1\leftarrow 1$\;
%     $i\leftarrow 1$\;
%     \While{$r_i\neq 0$}{
%         $q_i \leftarrow r_{i-1}\ \operatorname{quo}\ r_i$\;
%         $r_{i+1} \leftarrow r_{i-1}-q_i r_i$\;
%         $s_{i+1}\leftarrow s_{i-1}-q_i s_i$\;
%         $t_{i+1}\leftarrow t_{i-1}-q_i t_i$\;
%         $i\leftarrow i+1$\;
%     }
%     $l\leftarrow i-1$\;
%     \KwRet{$l,r_i, s_i, t_i$ for $0\leq i\leq l+1$, and $q_i$ for $1\leq i\leq l$.}
%     \caption{\textsc{Traditional Extended Euclidean Algorithm}}
% \end{algorithm}
% \end{minipage}
% \end{center}


% The purpose of introducing the variables $s_i$ and $t_i$ becomes clear when one notices that $s_i f + t_i g = r_i$ for all $1\leq i\leq l$. Note that : $s_i f + t_i g = r_i = r_{i+2}+q_{i+1}r_{i+1} = q_{i+1}(s_{i+1} f + t_{i+1}g) + r_{i+2}$. Hence, $r_{i+2} = (s_i-q_{i+1}s_{i+2})f + (t_i - q_{i+1}t_i)g$. Therefore, the following relations hold :
% \[s_{i+1}=s_{i-1}-q_i s_i, \quad t_{i+1}=t_{i-1}-q_i t_i. \tag{$ 0\leq i \leq l-1$}\]

% Let $f(X) = a_n X^n + a_{n-1} X^{n-1} + \ldots + a_{0} \in \mathbb{F}[X]$. Then $\tilde{f}(X) = X^n + a_n^{-1} a_{n-1} X^{n-1} + \ldots + a_n^{-1} a_0 \in \mathbb{F}[X]$ denotes the normalised version of $f(X)$. We will denote the leading coefficient of $f(X)$ by $\mathcal{L}(f)$. We rewrite our algorithm so that each division only involves monic polynomials. 

% \begin{center}
% \begin{minipage}{.75\linewidth}
% \begin{algorithm}[H]
%     \SetAlgoLined
%     \KwData{\(f,g\in \mathbb{F}[X]\), where $\mathbb{F}$ is a field, $\deg{f}\geq \deg{g}, g\neq 0$.}
%     \KwResult{\(l\in \N, \alpha_i, r_i,s_i,t_i\), for $0\leq i\leq l+1$ and $q_i\in\mathbb{F}[X]$ for $1\leq i\leq l$.}

%     $\alpha_0\leftarrow \mathcal{L}(f),\quad r_0\leftarrow f,\quad s_0\leftarrow 1,\quad t_0\leftarrow 0$\;
%     $\alpha_1\leftarrow \mathcal{L}(g),\quad r_1\leftarrow g,\quad s_1\leftarrow 0,\quad t_1\leftarrow 1$\;
%     $i\leftarrow 1$\;
%     \While{$r_i\neq 0$}{
%         $q_i \leftarrow r_{i-1}\ \operatorname{quo}\ r_i$\;
%         $\alpha_{i+1}\leftarrow \mathcal{L}(r_{i-1}-q_i r_i)$\;
%         $r_{i+1} \leftarrow(r_{i-1}-q_i r_i)\alpha_{i+1}$\;
%         $s_{i+1}\leftarrow (s_{i-1}-q_i s_i)\alpha_{i+1}$\;
%         $t_{i+1}\leftarrow (t_{i-1}-q_i t_i)\alpha_{i+1}$\;
%         $i\leftarrow i+1$\;
%     }
%     $l\leftarrow i-1$\;
%     \KwRet{$l,\alpha_i,r_i, s_i, t_i$ for $0\leq i\leq l+1$, and $q_i$ for $1\leq i\leq l$.}
%     \caption{\textsc{Extended Euclidean Algorithm}}
% \end{algorithm}
% \end{minipage}
% \end{center}

% We want all of the $r_i$'s to be monic. Therefore we write : $r_{i-1} = q_{i}r_{i}+\mathcal{L}(r_{i-1}-q_i r_i)^{-1}r_{i+1}$ and set $\alpha_{i+1} = \mathcal{L}(r_{i-1}-q_i r_i)^{-1}$. We still would like the relation : $s_i f + t_i g = r_i$ to hold. Therefore, similar to our previous calculation we can show that :
% \begin{gather*}
%     s_{i-1} f + t_{i-1} g = r_{i-1} = q_{i}r_{i} + \alpha_{i+1}^{-1}r_{i+1} = q_i (s_{i} f + t_i g) + \alpha_{i+1}^{-1}r_{i+1} \\
%     (s_{i-1}-q_i s_i) f + (t_{i-1}-q_i t_i) g = \alpha_{i+1}^{-1} r_{i+1} \\
%     s_{i+1} = \alpha_{i+1}(s_{i-1}-q_i s_i), \quad t_{i+1}=\alpha_{i+1}(t_{i-1}-q_i t_i)
% \end{gather*}

For $\mathfrak{N} = (n_0, n_1,\ldots, n_l)$ where $n_0\geq n_1$, consider the following map :
\begin{gather*}
    \psi_{\mathfrak{N}} : \mathscr{P}(\mathbb{F}_q, n_0)\times\mathscr{P}(\mathbb{F}_q, n_1) \to \mathscr{P}(\mathbb{F}_q, n_l)\bigtimes_{i=0}^{l-1} \mathscr{P}(\mathbb{F}_q, n_i-n_{i+1}) \\
    (f,g) \mapsto (r_l, q_1, \ldots, q_l)
\end{gather*}
\begin{enumerate}
    \item \underline{\textsc{Well Defined}} : Given the degree sequence $\mathfrak{N} = (n_0, n_1,\ldots, n_l)$ of $(f,g)$, since $r_0=f$ and $r_1 = g$, the degrees of $f,g$ are $n_0$ and $n_1$. We note the following :
\begin{align*}
    \deg{r_0} = \deg{q_1}+\deg{r_1} &, \quad \deg{r_1} = \deg{q_2}+\deg{r_2} \\
    &\vdots \\
    \deg{r_{l-1}} &= \deg{q_{l}}+\deg{r_l}
\end{align*}
Hence, we get that $\deg{q_i} = n_{i-1}-n_i$ for $1\leq i\leq l$. Therefore, $\psi_{\mathfrak{N}}$ is well defined.

\item \underline{\textsc{Injective}} : If $\psi_{\mathfrak{N}}(f, g) = \psi_{\mathfrak{N}}(f',g')$ then  $(r_l,q_1,\ldots, q_l) = (r_l', q_1',\ldots, q_l')$. Hence $r_{l-1}=q_l r_l = q_l' r_l' = r_{l-1}'$. Similarly, $r_{l-2}=q_{l-1}r_{l-1}+r_l = q_{l-1}'r_{l-1}'+r_l'=r_{l-2}'$. Continuing in this manner, we note that $f=r_0 = r_0'=f'$ and $g=r_1 = r_1'=g'$. Therefore, $\psi_{\mathfrak{N}}$ is an injective map.

\item \underline{\textsc{Surjective}} : Let $(r_l,q_1,\ldots, q_l)\in \mathscr{P}(\mathbb{F}_q, n_l)\bigtimes_{i=0}^{l-1} \mathscr{P}(\mathbb{F}_q, n_i-n_{i+1})$ then if we set $\begin{pmatrix}f \\ g\end{pmatrix} = \displaystyle\prod_{i=1}^l \begin{pmatrix}q_i & 1 \\ 1 & 0\end{pmatrix}\begin{pmatrix}r_l \\ 0\end{pmatrix}$ then it is clear that $\psi(f,g) = (r_l,q_1,\ldots, q_l)$. Therefore, $\psi_{\mathfrak{N}}$ is surjective.
\end{enumerate}
Therefore, $\psi_{\mathfrak{N}}$ is a bijection. If there exists a bijection between two finite sets, then their cardinalities must be the same. Therefore it suffices to calculate the cardinality of the set :
\[\mathfrak{S}_{\mathfrak{N}}:=\mathscr{P}(\mathbb{F}_q, n_l)\bigtimes_{i=0}^{l-1} \mathscr{P}(\mathbb{F}_q, n_i-n_{i+1}).\]

It follows that :
\begin{align*}
    |\mathfrak{S}_{\mathfrak{N}}| &= |\mathscr{P}(\mathbb{F}_q, n_l)\bigtimes_{i=0}^{l-1} \mathscr{P}(\mathbb{F}_q, n_i-n_{i+1})| \\
    &=|\mathscr{P}(\mathbb{F}_q, n_l)|\prod_{i=0}^{l-1}|\mathscr{P}(\mathbb{F}_q, n_i-n_{i+1})| \\
    &=(q-1)q^{n_l}\prod_{i=0}^{l-1}(q-1) q^{n_i-n_{i+1}} \\
    &=(q-1)q^{n_l}(q-1)^{l} q^{\sum_{i=0}^{l-1} n_i - n_{i+1}} \\
    &=(q-1)^{l+1}q^{n_l}q^{n_0-n_l} \\
    &=(q-1)^{l+1}q^{n_0}.
\end{align*}

Therefore, the number of pairs of polynomials $(f,g)$ such that $\deg{f}\geq\deg{g}$ and the degree sequence of $(f,g)$ is $(n_0,\ldots, n_l)$ is $(q-1)^{l+1}q^{n_0}$.

For degree sequence $(4,3,1,0)$, the number of pairs $(f,g)\in (\mathbb{F}_q[X]\setminus\{0\})^2$ is \textcolor{teal}{$(q-1)^4 q^4$}.

The only possible degree sequences for $n_0 = 3$ and $n_1 = 2$ are $(3,2,1,0)$ and $(3,2,0)$. These are the degree sequences for all possible ordered pairs $(f,g) \in \mathscr{P}(\mathbb{F}_2, 3)\times\mathscr{P}(\mathbb{F}_2, 2)$ : 

\begin{table}[h]
    \centering
    \begin{tabular}{|l||c|c|c|c|}
    \hline
           $(f,g)$ & $x^2$ & $x^2+1$ & $x^2+x$ & $x^2+x+1$\\
            \hline\hline
        $x^3$ &  $(3,2,0)$& $(3,2,1,0)$ & $(3,2,1,0)$ & $(3,2,0)$\\
        \hline
        $x^3+1$ & $(3,2,0)$ &$(3,2,1,0)$  & $(3,2,1,0)$ & $(3,2,0)$\\
        \hline
        $x^3+x$ &$(3,2,1,0)$  &  $(3,2,0)$& $(3,2,0)$ & $(3,2,1,0)$\\
        \hline
        $x^3+x+1$ & $(3,2,1,0)$ & $(3,2,0)$ & $(3,2,0)$ & $(3,2,1,0)$\\
        \hline
        $x^3+x^2$ & $(3,2,0)$ & $(3,2,1,0)$ & $(3,2,0)$ & $(3,2,1,0)$\\
        \hline
        $x^3+x^2+1$ & $(3,2,0)$ & $(3,2,1,0)$ & $(3,2,0)$ & $(3,2,1,0)$\\
        \hline
        $x^3+x^2+x$ & $(3,2,1,0)$ &  $(3,2,0)$& $(3,2,1,0)$ & $(3,2,0)$\\
        \hline
        $x^3+x^2+x+1$ & $(3,2,1,0)$ &  $(3,2,0)$& $(3,2,1,0)$ &$(3,2,0)$ \\
        \hline
    \end{tabular}
\end{table}