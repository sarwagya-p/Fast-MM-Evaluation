Note that : 
\[2^{2n}-1 = (2^n)^2 - 1 = (2^n - 1)(2^n+1) \equiv 0 \pmod{2^n + 1}\]
Therefore, $2$ is always a $2n$-th root of unity modulo $2^n + 1$.

\textbf{Necessity}
Assume that 2 is a $2n$-th primitive root of unity.
For the sake of contradiction, let $p$ be an odd prime that divides $n$. We can write $n$ as $pn'$, where $1\leq n \in \N$. Since $p$ is odd, we can write :
\[\forall z \in \Z_{2^n+1}: z^p + 1 = (z+1)(z^{p-1}-z^{p-2}+\ldots-z+1)\]
Substituting $z=2^{n'}$, we get : 
\[(2^{n'}+1)(2^{n'(p-1)}-2^{n'(p-2)}+\ldots-2^{n'}+1) = 2^{pn'}+1 = 2^n+1 \equiv 0 \pmod{2^n+1}\]
It follows that : 
\[(2^{n'}-1)(2^{n'}+1)(2^{n'(p-1)}-2^{n'(p-2)}+\ldots-2^{n'}+1)\equiv 0 \pmod{2^n + 1}\]
Hence, 
\[(2^{2n'}-1)(2^{n'(p-1)}-2^{n'(p-2)}+\ldots-2^{n'}+1)\equiv 0 \pmod{2^n + 1}\]
We observe that :
\begin{align*}
    0 < 2^{n'(p-1)}-2^{n'(p-2)}+\ldots-2^{n'}+1 &= 2^{n-n'}-2^{n-2n'}+\ldots - 2^{n-(p-1)n'}+1 \\
    &\le 2^{n-1} + 2^{n-2} \ldots + 1 \\
    &= 2^n - 1 \\
    \Rightarrow 2^{n'(p-1)}-2^{n'(p-2)}+\ldots-2^{n'}+1 &\not\equiv 0 \pmod{2^n+1}
\end{align*}
Hence, $2^{2n'}-1$ is a zero divisor in $\Z/(2^n+1)\Z$, which contradicts the primitiveness of $2$ in $\Z/(2^n+1)\Z$.

Therefore, no odd prime divides $n$.

\textbf{Sufficiency}

Let $n$ be a positive integer and $\omega\in R$.
\begin{enumerate}
    \item $\omega$ is a $n$-th root of unity if $\omega^n = 1$.
    \item $\omega$ is a primitive $n$-th root of unity if :
    \begin{enumerate}
        \item $\omega^n = 1$
        \item $n$ is a unit in $R$
        \item for every prime divisor $t$ of $n$, the element $\omega^{n/t}-1$ is neither zero nor a zero divisor.
    \end{enumerate}
\end{enumerate}

In our case we are working with $2n$-th roots of unity, $R = \Z/(2^n+1)\Z$ and $\omega = 2$. If $n=2^m$, where $m\in \Z_{\geq 1}$, then

$2n = 2^{m+1}$ is a unit in $\Z/(2^{2^m}+1)\Z$ because $\gcd(2^{m+1},2^{2^m}+1) = 1$. 

Since $2$ is the only prime that divides $n$, it suffices to show that $2^{2n/2}-1 = 2^n-1 = 2^{2^m}-1$ is neither zero not a zero divisor in $\Z/(2^{2^m}+1)\Z$. 

We first note that $2^{2^m}-1 \equiv -2 \pmod{2^{2^m}+1}$, so it is not $0$. Since $\gcd(2,2^{2^m}+1) = 1$, $2$ is a unit in $\Z/(2^n+1)\Z$ and therefore $-2\equiv 2^{2^m}-1$ is a unit in $\Z/(2^n+1)\Z$. Units in a ring can never be zero divisors, therefore $2$ is a primitive $2n$-th root of unity modulo $2^n+1$.\hfill $\blacksquare$.

