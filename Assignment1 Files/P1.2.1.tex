We first show that if $\omega$ is a $3^k$-th primitive root of unity for k > 0, then $\omega^3$ is a $3^{k-1}$-th primitive root of unity.

Since 3 is the only prime divisor of $3^{k-1}$, it suffices to show that:

\[
    (w^3)^{3^{k-1}/3} = w^{3^{k-1}}
    = w^{3^k/3} \ne 1
\]

As a corollary, $\exists 3-$th primitive root of unity. Let $\zeta = \omega^{3^{k-1}} = \omega^{n/3}$ be a 3-th primitive root of unity.

Let $f(x) = f_0 + f_1 x ... f_{n-1} x^{n-1}$

\begin{algorithm}[H]
    \SetAlgoLined
    \KwData{$f = (f_0, f_1, \cdots f_{n-1}), \omega$}
    \KwResult{$\Tilde{f} = \left(f(1), f(\omega), f(\omega^2) \cdots, f(\omega^{n-1})\right)$}
        \If{\(n = 1\)}{
        \Return{($f_0$)} 
    }
        
        $\Tilde{f}\leftarrow (0,0,\ldots, 0)$\tcp*{n 0's}

        $F_1 \leftarrow (f_0, f_3, \cdots, f_{n-3})$ \\
        $F_2 \leftarrow (f_1, f_4, \cdots, f_{n-2})$ \\
        $F_3 \leftarrow (f_2, f_5, \cdots, f_{n-1})$ \\[1\baselineskip]

        $\Tilde{F_1} \leftarrow \textsc{3-Adic-FFT}(F_1, \omega^3)$ \\
        $\Tilde{F_2} \leftarrow \textsc{3-Adic-FFT}(F_2, \omega^3)$ \\
        $\Tilde{F_3} \leftarrow \textsc{3-Adic-FFT}(F_3, \omega^3)$ \\[1\baselineskip]

        \For{$k\leftarrow 0$ \KwTo $n/3$}{
            $\Tilde{f}[k] \leftarrow \Tilde{F_1}[k] + \omega \Tilde{F_2}[k] + \omega^2 \Tilde{F_3}[k]$ \\
            $\Tilde{f}[k+n/3] \leftarrow \Tilde{F_1}[k] + \omega^k \zeta \Tilde{F_2}[k] + \omega^{2k} \zeta^2 \Tilde{F_3}[k]$ \\
            $\Tilde{f}[k+2n/3] \leftarrow \Tilde{F_1}[k] + \omega^k \zeta^2 \Tilde{F_2}[k] + \omega^{2k} \zeta \Tilde{F_3}[k]$ \\
        }
        
        \KwRet{$\Tilde{f}$}\;
    \caption{\textsc{3-Adic-FFT} Algorithm}
\end{algorithm}

\textbf{Proof of Correctness: } 

For $0 \le k < n: $

\begin{align*}
    f(\omega^k) &= f_0 + f_1 \omega^k + f_2 \omega^{2k} \cdots f_{n-1} \omega^{(n-1)k} \\
    &= (f_0 + f_3 \omega^{3k} \cdots f_{n-3} \omega^{(n-3)k}) + (f_1 \omega^k + f_4 \omega^{4k} \cdots f_{n-2} \omega^{(n-2)k}) + (f_2 \omega^{2k} + f_5 \omega^{5k} \cdots f_{n-1} \omega^{(n-1)k}) \\
    &= (f_0 + f_3 \omega^{3k} \cdots f_{n-3} \omega^{(n-3)k}) + \omega^k (f_1 + f_4 \omega^{3k} \cdots f_{n-2} \omega^{(n-3)k}) + \omega^{2k} (f_2 + f_5 \omega^{3k} \cdots f_{n-1} \omega^{(n-3)k})
\end{align*}

Now, for $n/3 \le k < 2n/3$: Let $k' = k - n/3$

\begin{align*}
    \omega^k &= \omega^{k' + n/3} = \zeta \omega^{k'} \\
    \omega^{2k} &= \omega^{2k' + 2n/3} = \zeta^2 \omega^{2k'}
\end{align*}

Similarly, for $2n/3 \le k < n$: Let $k' = k - 2n/3$

\begin{align*}
    \omega^k &= \omega^{k' + 2n/3} = \zeta^2 \omega^{k'} \\
    \omega^{2k} &= \omega^{2k' + 4n/3} = \zeta^4 \omega^{2k'} = \zeta \omega^{2k'}
\end{align*}

Hence, 
\[ f(\omega^k) = 
\begin{cases}
    F_1(\omega^k) + \omega^k F_2(\omega^k) + \omega^{2k} F_3(\omega^k) & 0 \le k < n/3 \\
    F_1(\omega^{k'}) + \omega^{k'} \zeta F_2(\omega^{k'}) + \omega^{2k'} \zeta^2 F_3(\omega^{k'}) & n/3 \le k < 2n/3, k' = k - n/3 \\
    F_1(\omega^{k'}) + \omega^{k'} \zeta^2 F_2(\omega^{k'}) + \omega^{2k'} \zeta F_3(\omega^{k'}) & n/3 \le k < n, k' = k - 2n/3
\end{cases}
\]

\textbf{Time Complexity: } Let $T(n)$ be the worst case time taken for input of length $n$, where $n$ is a power of $3$.

Line 4-7 of algorithm are shifting and thus takes $O(n)$ time. Line 8-10 is recursive call with input of size $n/3$, and thus takes $3 T(n/3)$ time in worst case, while like 11-15 is a for loop with $n/3$ iterations, each with a constant number of operations (in $R$) and thus takes $O(n)$ time.

Since lines 4-7 and 11-15 take $O(n)$ time, let $c \in \R^+$ such that time taken by line 4-7 and 11-15 $\le cn$. Moreover, let $k \in \N: n = 3^k$. Therefore we get the recursion:

\begin{align*}
    T(n) &\le 3 T\left(\frac{n}{3}\right) + cn \\
    \Rightarrow T(3^k) &\le 3 T(3^{k-1}) + c 3^k \\
\end{align*}

Unravelling the recursion, we get

\begin{align*}
    T(3^k) & \le 3^k T(1) + c \left(1 \times 3^k + 3 \times 3^{k-1} + 3^2 \times 3^{k-2} \cdots 3^k \times 1\right) \\
    &= 3^k T(1) + ck \cdot 3^k \\
    \Rightarrow T(n) &\le c'n + cn \log_3(n) \\
    \Rightarrow T(n) &= O(n \log n)
\end{align*}