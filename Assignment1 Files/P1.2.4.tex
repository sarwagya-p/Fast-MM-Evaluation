First, we claim that for any $l(>t)$ bit integer $z \in \Z$, we can get a t ($>0$) bit floating point approximation of $z$ by rounding to the nearest multiple of $2^{l-t}$ such that:

\[
\hat{z} = a \cdot 2^e, a, e \in \Z,  \log_2{|a|} = t \text{ such that } \exists \epsilon: \hat{z} = (1+\epsilon)z, |\epsilon| \le 2^{-t} 
\]

Note that error is $2^{-t}$ instead of $2^{-t+1}$, since we have the extra flexibility of rounding up or down. The error bound can be proven as:

Let $z = b_{l-1}b_{l-2} \cdots b_1 b_0$, where $\forall i: b_i \in \Z_2$ and $b_0$ is LSB.

Define \(\hat{z} =a 2^e, \text{ where } a = b_{l-1}b_{l-2} \cdots b_{l-t} + b_{l-t-1} \text{ and } e = l-t\) (Note: in the edge case if addition in $a$ overflows to $t+1$ bits, then redefine $a = 100...0, e = l-t+1$)

Then, 
\begin{align*}
    \frac{z - \hat{z}}{z} &= \frac{2^{l-t}b_{l-t-1} - b_{l-t-1} \cdots b_0}{z} \\
    &= \begin{cases}
        - b_{l-t-2} \cdots b_0 & b_{l-t-1} = 0 \\
        2^{l-t-1} - b_{l-t-2} \cdots b_0 & b_{l-t-1} = 1
    \end{cases} \\
    &\le \frac{2^{l-t-1}}{z} \\
    &\le \frac{2^{l-t-1}}{2^{l-1}} \le 2^{-t}
\end{align*}

Now, given $b, t \in Z$, we claim that we can find a 4 bit approximation of 1/b, and for $t \ge 4$ if we have a $t$-bit floating point approximation of $1/b$, then we can generate a $2t-3$ bit approximation.

Define $f(x) = \frac{1}{x} - b$. Then, we want to find the root of this function. Thus we can use the newton's method update rule:
\begin{align*}
    f'(x) &= -\frac{1}{x^2} \\
    x - \frac{f(x)}{f'(x)} &= x + (x - bx^2) = x(2 - bx) \\
\end{align*}
Update Rule: $x_{n+1} = x_n(2 -bx_n)$

We can find a 4 bit approximation of 1/b in constant time (Consider an 8 bit approx of b, and take the 4bit float which is closest to 1).

Now, for $t \ge 3$, let $x_t = a_t 2^{e_t}$ be a $t$ bit approximation of $1/b$, and $b_{2t+1}$ be a $(2t+1)$ bit approximation of integer $b$. Then, 
\begin{align*}
    \exists \epsilon \in \R: x_t &= \frac{1+\epsilon}{b}, |\epsilon| \le 2^{-t+1} \\
    \exists \epsilon' \in \R: b_{2t+1} &= (1+\epsilon'), |\epsilon'| \le 2^{-2t-1}
\end{align*}

Define $\hat{x}$ and $\epsilon''$ as:
\begin{align*}
    \hat{x} &= x_t (2 - b_{2t+1} x_t) \\
    &= \frac{(1+\epsilon)}{b}\left(2 - (1+\epsilon')b \cdot \frac{(1+\epsilon)}{b}\right) \\
    &= \frac{(1+\epsilon)}{b} \left(1 - \epsilon - \epsilon' - \epsilon \epsilon'\right) \\
    &= \frac{1}{b} \left(1 - \epsilon' - 2\epsilon \epsilon' - \epsilon^2 - \epsilon^2\epsilon'\right) \\
    &= \frac{1}{b}\left(1 - \epsilon^2 - \epsilon'(1+\epsilon)^2\right) = \frac{1+\epsilon''}{b}
\end{align*}

For \(t \ge 3, |\epsilon| \le 2^{-2} = 0.25 \Rightarrow (1+\epsilon)^2 \le 2\). Thus:

\begin{align*}
    |\epsilon''| \le 2^{-2t+2} + 2^{-2t-1} \cdot 2 \le 1.25 \cdot 2^{-2t+2} \le 2^{-3}
\end{align*}

Now, we take a $(2t-3)$ bit approximation of $\hat{x}$, $x_{2t-3}$.

Then, \(\exists \hat{\epsilon} \in \R: x_{2t-3} = \hat{x}(1+\hat{\epsilon}), |\hat{\epsilon}| \le 2^{-2t+3}\)

\begin{align*}
    \left|\frac{1/b-x_{2t-3}}{1/b}\right| &= |1 - b \cdot x_{2t-3}| \\
    &= \left|1 - b \cdot \frac{1+\epsilon''}{b} \cdot (1+\hat{\epsilon})\right| \\
    &= |\hat{\epsilon} + \epsilon'' + \epsilon'' \hat{\epsilon}| \\
    &\le 2^{-2t+3}(1 + 2^{-3}) + 1.25 \cdot 2^{-2t+2} \\
    &= 2^{-2t+3}(1+ 1.25/2 + 2^{-3}) \\
    &\le 2^{-2t+4}
\end{align*}

Therefore, $x_{2t-3}$ is a valid $2t-3$ bit approximation of $1/b$.

This procedure take $O(M(t))$ time (since $b_t, x_t$ are $t$ bit floats), and can be applied repeatedly till we have an k bit approximation of $1/b$, which can thus be found in $M(k) + M(k/2) + M(k/4) + \ldots + M(1) = O(M(k))$ time.

Now we argue hardness of division. Let $a, b \in \Z$ be two $l$-bit integers. We can find \(\lfloor a/b \rfloor\) and \(a \bmod b\) by following steps:

\begin{itemize}
    \item Compute $z$ = $l+3$ bit approximation of $1/b$.
    \item Compute $\hat{q} = a \cdot z$. Take nearest integer $q = \lfloor \hat{q} \rceil$. 
    \item Compute $\hat{r} = a - b \cdot q$. If: 
    \begin{itemize}
        \item $r \le b$: return $\lfloor a/b \rfloor = q$, $a \bmod b = r$.
        \item $r > b$: return $\lfloor a/b \rfloor = q+1$, $a \bmod b = r-b$.
    \end{itemize}
\end{itemize}

Note that:
\begin{align*}
    |q - a/b| &\le |\hat{q} - a/b + 1/2| \\
    &= |(1+\epsilon) a/b - a/b + 1/2| &\text{ (where $\epsilon$ is error due to approximation of $b$)} \\
    &\le 2^{-l-2} \cdot 2^l + 1/2 &\text{(since we took $l+3$ bit approx)} \\
    &< 1
\end{align*}

This means that:
\begin{align*}
    a/b - 1 &< q < a/b + 1 \\
    \Rightarrow \lfloor a/b \rfloor - 1 &< q < \lfloor a/b \rfloor + 2&\text{(since $x -1 < \lfloor x \rfloor \le x$)} \\
    \Rightarrow q - 1 &< \lfloor a/b \rfloor < q + 2
\end{align*}
Thus, the only possibilities of $\lfloor a/b \rfloor$ is $q$ or $q+1$ which is captured in the two cases of step 3.

The 3 steps all take $O(M(l))$ time, and thus integer division with remainder is no harder than integer multiplication upto a constant factor.